%!TEX TS-program = xelatex
%!TEX encoding = UTF-8 Unicode
% Awesome CV LaTeX Template for CV/Resume
%
% This template has been downloaded from:
% https://github.com/posquit0/Awesome-CV
%
% Author:
% Claud D. Park <posquit0.bj@gmail.com>
% http://www.posquit0.com
%
% Template license:
% CC BY-SA 4.0 (https://creativecommons.org/licenses/by-sa/4.0/)
%


%-------------------------------------------------------------------------------
% CONFIGURATIONS
%-------------------------------------------------------------------------------
% A4 paper size by default, use 'letterpaper' for US letter
\documentclass[11pt, a4paper]{awesome-cv}

% Configure page margins with geometry
\geometry{left=1.4cm, top=.8cm, right=1.4cm, bottom=1.8cm, footskip=.5cm}

% Specify the location of the included fonts
\fontdir[fonts/]

% Set false if you don't want to highlight section with awesome color
\setbool{acvSectionColorHighlight}{false}

% If you would like to change the social information separator from a pipe (|) to something else
\renewcommand{\acvHeaderSocialSep}{\quad\textbar\quad}

%-------------------------------------------------------------------------------
%	PERSONAL INFORMATION
%	Comment any of the lines below if they are not required
%-------------------------------------------------------------------------------

\name{Jean-Claude}{Desrosiers}
\position{Passionné, enseignant et étudiant en informatique}

\email{jean-claude.desrosiers@jclaude.xyz}
\github{jeanclaude-desrosiers}
\homepage{www.jclaude.xyz}
\mobile{(+1) 438-990-4360}
\linkedin{jclaude-des}
\gitlab{DesJC}

\usepackage{lastpage}

%-------------------------------------------------------------------------------
\begin{document}

% Print the header with above personal informations
% Give optional argument to change alignment(C: center, L: left, R: right)
\makecvheader[C]

% Print the footer with 3 arguments(<left>, <center>, <right>)
% Leave any of these blank if they are not needed
\makecvfooter
  {\the\year}
  {Jean-Claude Desrosiers~~~·~~~CV}
  {\thepage\ de \pageref{LastPage}}

%-------------------------------------------------------------------------------
%	CV/RESUME CONTENT
%	Each section is imported separately, open each file in turn to modify content
%-------------------------------------------------------------------------------

%-------------------------------------------------------------------------------
%	SECTION TITLE
%-------------------------------------------------------------------------------
\cvsection{Profile}


%-------------------------------------------------------------------------------
%	CONTENT
%-------------------------------------------------------------------------------
\begin{cvskills}

%---------------------------------------------------------
  \cvskill
    {Communication} % type
    {Esprit de travail d'équipe et de relation avec les clients; en français ou en anglais}
    
  \cvskill
    {Apprentissage} % type
    {Habileté d'apprentissage rapide des concepts et compétences en informatique}
    
  \cvskill
    {Travaillant} % type
    {Effort à produire du travail de qualité, efficacement}
    
  \cvskill
    {Curiosité} % type
    {Intérêt en programmation bas niveau, informatique théorique, informatique quantique, etc.}

%---------------------------------------------------------
\end{cvskills}


%-------------------------------------------------------------------------------
%	SECTION TITLE
%-------------------------------------------------------------------------------
\cvsection{Skills}


%-------------------------------------------------------------------------------
%	CONTENT
%-------------------------------------------------------------------------------
\begin{cvskills}

%---------------------------------------------------------
  \cvskill
    {Programming} % type
    {Fluency with multiple paradigms (object-oriented, declarative, functional, logical, etc.)}

  \cvskill
    {System} % type
    {Knowledgeable in low-level (C et MIPS) and machine (VHDL) programming}

  \cvskill
    {Development} % type
    {Experienced in designing and developing Java desktop applications with Maven}
    
  \cvskill
    {Web} % type
    {Knowledgeable in front-end (Bootstrap, Javascript) and back-end (NodeJS, PHP, .NET) development}
  
  \cvskill
    {Mobile} % type
    {Practical experience developing Android and React.js applications for mobile}
    
  \cvskill
    {Database} % type
    {Efficient in SQL query writing and database design}
    
  \cvskill
    {Linux} % type
    {Knowledgeable at using a Linux operating system and command-line utilities}

%---------------------------------------------------------
\end{cvskills}


%-------------------------------------------------------------------------------
%	SECTION TITLE
%-------------------------------------------------------------------------------
\cvsection{Education}


%-------------------------------------------------------------------------------
%	CONTENT
%-------------------------------------------------------------------------------
\begin{cventries}

%---------------------------------------------------------
  \cventry
    {Bachelor in Computer Science} % Degree
    {UdeM (Université de Montréal)} % Institution
    {Montreal, Quebec} % Location
    {Sept. 2019 - May 2022} % Date(s)
    {
      \begin{cvitems} % Description(s) bullet points
        \item {Practical and theoretical problems related to software design and development}
        \item {Awarded two scholarships of excellence in the computer science department in 2020 and 2022}
      \end{cvitems}
    }
  \cventry
    {Diploma of Collegial Studies in Computer Science} % Degree
    {Dawson College} % Institution
    {Montreal, Quebec} % Location
    {Sept. 2016 - May. 2019} % Date(s)
    {
      \begin{cvitems} % Description(s) bullet points
        \item {Rapid Application Development for desktop, web and mobile environments according to industry standards}
      \end{cvitems}
    }

%---------------------------------------------------------
\end{cventries}


%-------------------------------------------------------------------------------
%	SECTION TITLE
%-------------------------------------------------------------------------------
\cvsection{Work Experience}


%-------------------------------------------------------------------------------
%	CONTENT
%-------------------------------------------------------------------------------
\begin{cventries}

%---------------------------------------------------------  
  
  \cventry
    {Software Developer} % Job title
    {Desjardins} % Organization
    {Montreal, Quebec} % Location
    {May 2021 -} % Date(s)
    {
      \begin{cvitems} % Description(s) of tasks/responsibilities
        \item {Maintenance, exploitation and development with applications such as uCMDB, UD, SmartTrack and BigFix}
        \item {Use Windows batch scripting to automate jobs and Python for development}
        \item {Apply ITSM processes in the ITIL v.4 framework}
      \end{cvitems}
    }
    
  \cventry
    {IT Support Technician} % Job title
    {Desjardins} % Organization
    {Montreal, Quebec} % Location
    {May 2020 - May 2021} % Date(s)
    {
      \begin{cvitems} % Description(s) of tasks/responsibilities
        \item {Provide technical support for those working in Desjardins caisses}
        \item {Work in a team, along other teams, towards a common goal}
      \end{cvitems}
    }
    
  \cventry
    {Software Developer Intern} % Job title
    {Bacon Financial Technologies (Hardbacon)} % Organization
    {Longueuil, Quebec} % Location
    {Apr. 2019 - May 2019} % Date(s)
    {
      \begin{cvitems} % Description(s) of tasks/responsibilities
        \item {Develop and designing (with a team) a mobile application for portfolio management and financial counseling}
        \item {Meet deadlines for releases of versions matching the marketing campaign}
      \end{cvitems}
    }

%---------------------------------------------------------
\end{cventries}


%-------------------------------------------------------------------------------
%	SECTION TITLE
%-------------------------------------------------------------------------------
\cvsection{Activités Extra-Scolaires}


%-------------------------------------------------------------------------------
%	CONTENT
%-------------------------------------------------------------------------------
\begin{cventries}

%---------------------------------------------------------
  \cventry
    {Bénévole} % Affiliation/role
    {Code Club Canada} % Organization/group
    {St-Philippe, Québec} % Location
    {Sept. 2019 - Avril 2021} % Date(s)
    {
      \begin{cvitems} % Description(s) of experience/contributions/knowledge
        \item {Planifier une série d'activitées Scratch, pour l'apprentissage ludique des base de la programmation}
        \item {Animer ces activités, hebdomadaires, avec un petit groupe d'enfants}
      \end{cvitems}
    }

%---------------------------------------------------------
\end{cventries}


%-------------------------------------------------------------------------------
\end{document}
